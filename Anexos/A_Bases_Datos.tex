\chapter{Bases de datos utilizadas}

Resultados electorales completos: 

La prelimpieza de estas bases solo consistió en eliminar los códigos al inicio de los archivos de texto. Todos tienen el mismo esquema de nombre original $EEaa\_Bvot\_T1T2$ donde EE es LG para legislativas y PR para presidenciales y los años son 07 y 12. Las bases alteradas tienen el sufijo FAZH, separado por guión bajo. 
\begin{enumerate}
\item Presidenciales 2007 \parencite{PresFr07}
\item Legislativas 2007 \parencite{LegisFr07}
\item Presidenciales 2012 \parencite{PresFr12}
\item Legislativas 2012 \parencite{LegisFr12}
\end{enumerate}

Hasta el momento tengo las siguientes bases de datos: 

\begin{itemize}
\item \textbf{Datos censales}, a nivel comuna que incluyen 3 grupos de edad comparables, 2 categorías de situación migratoria, 2 categorías de nacionalidad, sexo, 6 categorías de tipo de actividad y 8 categorías socioprofesionales. 
\begin{enumerate}
\item IMG1 2007
\item NAT3A 2007
\item IMG2A 2012
\item NAT3A 2012
\end{enumerate}
\item \textbf{Datos de escolaridad}, a nivel comuna y que incluyen grupos de edad y por sexo para personas escolarizadas y no escolarizadas. De las no escolarizadas se tiene el grado más elevado alcanzado, 7 categorías.
\begin{enumerate}
\item Diplômes-Formation 2007
\item Diplômes-Formation 2012
\end{enumerate}
\item \textbf{Datos económicos}, a nivel (algunas) comuna, departamento y región.
\begin{enumerate}
\item Algunos datos de distribución de ingresos por comunas, con secreto estadístico
\item Distribución del ingreso y Gini por dpto.
\item PIB real por región.
\end{enumerate} 
\item \textbf{Datos delictivos}, a nivel departamento, casos por diferentes tipos de delitos (víctimas, procedimientos, instancias, agresores)
\begin{enumerate}
\item Datos mensuales 1996-2017
\end{enumerate}
\end{itemize}

Sin embargo tengo algunas dudas: 

\begin{enumerate}
\item ¿Cómo mezclar los datos delictivos con diferentes unidades de medición? 
\item Hice una sublista de delitos que consideré poco relevantes, para excluirlos del análisis. 
\item Dado que los datos delictivos son mensuales, ¿debería usar datos del año de la elección o de los 12 meses anteriores a la elección?
\item Los datos económicos tuvieron algunos cambios entre los periodos de análisis. Para datos de distribución de ingreso y Gini se pueden hacer comparaciones entre 2006 y 2011, pero no de 2007 a 2012. Para datos del PIB cambió la forma de medirlo en 2008. 
\item En general, ¿combiene usar niveles o cambios?
\item Para el AED, ¿presento variable por variable o de una vez cruces?
\end{enumerate}

\begin{itemize}
\item Para delitos, tasas por cada 100,000 hab en tres categorías (baja, media, alta) y caja y brazos. 
\item Si tiene que ser proporciones. Puede ser modelo binomial para cada partido con n el listado nominal. Normal para el logit de las prop desplazadas. 
\end{itemize}

\begin{table}
\centering
\begin{tabular}{l c c c}
\textbf{Variable} & \textbf{Tipo} & \textbf{2007} & \textbf{2012} \\
\hline 
Vuelta & C & $\diamond$ & $\diamond$  \\
Código de departamento & C & $\diamond$ & $\diamond$  \\
Código de comuna & C & $\diamond$ & $\diamond$  \\
Nombre de comuna & C & $\diamond$ & $\diamond$  \\
Circunscripción legislativa & C &  & $\diamond$  \\
Número de cantón & C &  & $\diamond$  \\
Número de casilla & C & $\diamond$ & $\diamond$  \\
Inscritos & N & $\diamond$ & $\diamond$  \\
Votantes & N & $\diamond$ & $\diamond$  \\
Votos expresados & N & $\diamond$ & $\diamond$  \\
Número de candidatura & C & $\diamond$ & $\diamond$  \\
Apellido de candidato(a) & C & $\diamond$ & $\diamond$  \\
Nombre(s) de candidato(a) & C & $\diamond$ & $\diamond$  \\
Etiqueta política & C & $\diamond$ & $\diamond$  \\
Votos para la candidatura  & N & $\diamond$ & $\diamond$  \\
\end{tabular}
\caption{Descripción de variables en las bases de resultados electorales a nivel casilla para las elecciones presidenciales y legislativas de 2007 y 2012. Los tipos de variables son categóricas (C) o numéricas (N). El diamante indica que la variable está presente en las bases correspondientes. Fuente: elaboración propia.}
\label{tbl:Bases_electorales}
\end{table}


\begin{table}
\centering
\resizebox{\linewidth}{!}{
\begin{tabular}{l *{5}{c}}
\textbf{Variable} & \textbf{Tipo} 
& \textbf{IMG1A 2007} 
& \textbf{IMG2A 2012} 
& \textbf{NAT3A 2007}
& \textbf{NAT3A 2012}\\
\hline 
Nivel & C & $\diamond$ & $\diamond$ & $\diamond$ & $\diamond$  \\
CODGEO & C & $\diamond$ & $\diamond$ & $\diamond$ & $\diamond$  \\
LIBGEO & C &  & $\diamond$ & & $\diamond$ \\
Sexo & C & $\diamond$ & $\diamond$  & $\diamond$ & $\diamond$\\
Edad 4 & C & $\diamond$ &  &  & \\
Edad 4A & C &  & $\diamond$ &  &  \\
Tipo de actividad & C & $\diamond$ & $\diamond$ &  &  \\
Categoría socioprofesional & C &  &  & $\diamond$ & $\diamond$ \\
Condición migratoria & C & $\diamond$ & $\diamond$ &  &  \\
Nacionalidad & C &  &  & $\diamond$ & $\diamond$  \\
Número de personas & N & $\diamond$ & $\diamond$ & $\diamond$ & $\diamond$ \\
\end{tabular}
}
\caption{Descripción de variables en las bases censales a nivel comuna sobre migración y nacionalidad. Los tipos de variables son categóricas (C) o numéricas (N). El diamante indica que la variable está presente en la base correspondiente. Fuente: elaboración propia.}
\label{tbl:Bases_censales}
\end{table}

\begin{table}
\centering
\resizebox{\linewidth}{!}{
\begin{tabular}{l *{4}{c}}
\textbf{Variable} 
& \textbf{Datos administrativos}
& \textbf{Totales} 
& \textbf{Escolarizados} 
& \textbf{No Escolarizados de 15 años o más}\\
\hline 
CODGEO & $\diamond$ & $\diamond$ & $\diamond$ & $\diamond$\\
Código de Región & $\diamond$ & & & \\
Código de Departamento & $\diamond$ & & & \\
LIBGEO & $\diamond$ & & & \\
Sexo & & $\diamond$ & $\diamond$  & $\diamond$ \\
Edad 7 & & $\diamond$ & $\diamond$ & \\
Grado máximo de estudios & & & & $\diamond$ \\
\end{tabular}
}
\caption{Descripción de variables en las subbases a nivel comuna sobre escolaridad. El diamante indica que la variable está presente en la subbase correspondiente. Todas las variables son las mismas para 2007 y para 2012. Fuente: elaboración propia.}
\label{tbl:Bases_escolaridad}
\end{table}

\begin{table}
\centering
\resizebox{\linewidth}{!}{
\begin{tabular}{l *{4}{c}}
\textbf{Variable} 
& \textbf{Datos administrativos}
& \textbf{Totales} 
& \textbf{Escolarizados} 
& \textbf{No Escolarizados de 15 años o más}\\
\hline 
CODGEO & $\diamond$ & $\diamond$ & $\diamond$ & $\diamond$\\
Código de Región & $\diamond$ & & & \\
Código de Departamento & $\diamond$ & & & \\
LIBGEO & $\diamond$ & & & \\
Sexo & & $\diamond$ & $\diamond$  & $\diamond$ \\
Edad 7 & & $\diamond$ & $\diamond$ & \\
Grado máximo de estudios & & & & $\diamond$ \\
\end{tabular}
}
\caption{Descripción de variables en las subbases a nivel comuna sobre escolaridad. El diamante indica que la variable está presente en la subbase correspondiente. Todas las variables son las mismas para 2007 y para 2012. Fuente: elaboración propia.}
\label{tbl:Bases_escolaridad}
\end{table}