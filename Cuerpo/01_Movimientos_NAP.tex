\chapter{Los movimientos NAP}

En 2017, el diccionario Cambridge declaró al \textit{populismo} como la palabra del año \parencite{MuddeCambridge17}. Se percibiría entonces una \textit{ola populista} a nivel mundial como uno de los temas más interesantes para estudiar sobre política internacional. El triunfo de Donald Trump en las elecciones de 2016 en Estados Unidos supuso, al menos en términos de percepción, su expansión y manifestación más importante no solo por la relevancia que reviste la nación norteamericana sino también por la atención mediática que generó en todo el mundo, las posibles consecuencias para el sistema internacional, el Estado y las vidas concretas de miles de seres humanos.\\

Ahora bien, Trump no es el único ni el primero de estos populistas. Los movimientos y partidos políticos de esta ola son altamente variables. La primera diferencia entre ellos es la cantidad de apelativos con la que son identificados. Es posible encontrar etiquetas tan variadas como \textit{nacionalismo populista}, \textit{tribalismo reaccionario}, \textit{neopopulismo} o, incluso, \textit{neofascismo}, \textit{postfascismo} o \textit{neonazismo} \parencites{Mudde07a}{Mammone12}{Hainsworth16a}; etiquetas que, en la mayoría de las ocasiones, los propios partidos rechazan \parencites{LeParisien13}{Hainsworth16a}{Sputnik17}.\\ 

Independientemente del debate sobre el término correcto, en general hay un acuerdo sobre cuáles movimientos pertenecen al núcleo de esta familia política. Esto permite ilustrar otra de sus características altamente variable: la geografía. Como mencionaba en la introducción, es un partido europeo--- el Front National francés--- el referente de esta familia política \parencite{Mudde07a}. Además, podemos mencionar dentro de la categoría a varios partidos que han tenido éxitos electorales recientes y que motivaron la elección del populismo como palabra del año: el UKIP de Nigel Farage como impulsor del Brexit; la Lega de Matteo Salvini en Italia, socio principal del gobierno del M5S; el FPÖ astríaco, formando parte de la coalición gobernante surgida de las parlamentarias de 2017; el PVV holandés de Geert Wilders, quien llegó a ser líder en las encuestas hacia la elección de 2017 o la AfD alemana, misma que consiguió lugares en el Bundestag, gracias a sus buenos resultados particularmente en el este del país.\\ 

Otros ejemplos de partidos de esta corriente en regiones europeas tales como Escandinavia, Europa del Este o los Balcanes se pueden consultar en \textcite{Mammone12} o \textcite{Hainsworth16a}. Pero, como sugieren el hablar de una \textit{ola populista} o el triunfo de Trump, el fenómeno no es exclusivamente europeo, pues se da en partidos de Australia o Indonesia y líderes en el poder como Narendra Modi en India, Jair Bolsonaro en Brasil, Rodrigo Duterte en Filipinas o Benjamin Netanyahu en Israel.\\ 

A su vez, dicha extensión geográfica muestra que estos movimientos se pueden encontrar en países institucionalmente muy disímiles. Los hay bajo monarquías parlamentarias, sistemas federales o regímenes más centralizados; con elecciones de mayoría relativa, de representación proporcional o mixtas; con una o dos vueltas. Otra característica cambiante es que tampoco han estado exentos de evoluciones temporales, ni son un fenómeno completamente nuevo. Mientras que en los años 80 la mayoría de estos partidos tenían un programa económico marcadamente favorable al libre mercado, en los años recientes han migrado hacia posiciones mucho más proteccionistas \parencite{Mudde16}.\\ 

A pesar de estas diferencias, es posible entender al fenómeno como una sola familia política trasnacional. Aunque hay que reconocer que, incluso si existen referencias que estudian su desarrollo fuera de Europa --- por ejemplo,  \textcite{CoxDurham16}--- la realidad es que las caracterizaciones tradicionales que uno puede encontrar en la literatura frecuentemente se basan en elementos que aluden explícitamente al viejo continente.\\ 

Por ejemplo, \textcite{Mammone12}, siguiendo al sociólogo Alain Bihr, señalan dos aspectos que consideran fundamentales, aunque aparentemente contradictorios. Con la creciente globalización, los Estados-Nación se percibirían como inoperantes e impotentes ante las decisiones tomadas “en otro lado, por otros”; por ejemplo, en Bruselas por la Unión Europea. En consecuencia, estos partidos buscan resaltar a la Nación. Sin embargo, al mismo tiempo añoran o reclaman a Europa como hogar. Esta sería una Europa blanca, opuesta a la globalización, la hegemonía de EUA, la pluralidad étnica y el Islam. Así, \citeauthor{Mammone12} encuentran que los partidos han buscado dejar el simple nacionalismo para hablar de un “nacionalismo europeo”.\\ 

Por otro lado, estos movimientos podrían analizarse como una familia política porque comparten, según los autores, tres pilares básicos: la idea de la inequidad y la jerarquía, un nacionalismo étnico vinculado a la comunidad mono racial y la adopción de medios radicales para lograr sus objetivos y defender a la comunidad imaginada. De entre estos aspectos el más claro y relevante podría ser el etnonacionalismo. Los inmigrantes--- no blancos y, en general, provenientes de ex colonias--- serían vistos como poseedores de culturas diferentes que atentan contra la milenaria tradición nacional. Los autores consideran esta postura diferencialista como simple racismo pues cambiar el término biológico por el cultural es solo retórica.\\

Esta interpretación de que la retórica culturalista es más bien racismo cultural es compartida por muchos autores. \textcite{Goodliffe17}, en un foro sobre el crecimiento del populismo en el mundo, también señaló que las teorías e ideas diferencialistas son, en el fondo, manifestaciones de racismo.  \textcite{Hainsworth16a} refiere la propia visión al tiempo que presenta varias citas que apoyan esta argumentación. Por lo mismo, y sin que quede claro si el motivo es solamente aparentar ser una opción más aceptable, estos partidos habrían buscado bajar el tono de dicha retórica sustituyéndolo por un abierto populismo. El mismo \textcite{Hainsworth16a}, refiere otra serie de elementos que pudieran caracterizar a esta familia política: chauvinismo de bienestar, ley y orden, la búsqueda de un Estado fuerte, el carácter antisistema o incluso la antidemocracia, entre otros.\\ 

A pesar de coincidir en algunas de estas características, no parece ser en movimientos como Syriza de Alexis Tsipras o Podemos de Pablo Iglesias en los que los medios y académicos internacionales están pensando cuando hablan de ola populista. Tampoco se piensa en populismos latinoamericanos como los de los países del ALBA ni en Morena de López Obrador. Estos cuatro ejemplos serían considerados populistas de izquierda \parencite{MuddeRovira17}. Por el contrario, el mayor interés está en aquellos movimientos que podríamos llamar \textit{populismos de derecha}, \textit{derechas extremas} o \textit{derechas radicales}.\\

Por lo mismo, para poder tener un mayor entendimiento teórico del fenómeno se debe partir de una delimitación más puntual y formal de las características que hacen a un partido determinado pertenecer a esta familia política. Se debe pues, \textit{definir} teóricamente a la familia política. Éste es el objetivo de Cas \textcite{Mudde07a}. 

\section{Definiendo una familia política}

\citeauthor{Mudde07a} busca encontar la máxima cantidad de similitudes que puedan tener los partidos normalmente asociados al fenómeno para pertencer a esta familia política. Así, logra construir de manera simple pero estructurada una caracterización teórica de lo que él llama los partidos de derecha radical de corte populista. Para ello, hace uso, principalmente, de tres definiciones básicas que presento a continuación.

\begin{itemize}
\item \textbf{Nativismo}\\
El \textit{nativismo} es una ideología que mantiene que los Estados deberán ser habitados exclusivamente por miembros de un grupo nativo--- la Nación--- y que los elementos no nativos--- personas e ideas--- amenazan fundamentalmente el Estado-Nación homogéneo.

\item \textbf{Autoritarismo}\\
El \textit{autoritarismo} es una creencia en una sociedad estrictamente ordenada en la cual las violaciones a la autoridad deben ser castigadas severamente.

\item \textbf{Populismo}\\
El \textit{populismo} es una ideología estrechamente centrada que considera que la sociedad está, al fin y al cabo, separada en dos grupos homogéneos al interior y antagónicos entre sí--- ``el pueblo puro'' vs ``la élite corrupta''--- y que argumenta que la política debe ser una expresión de la voluntad general del pueblo.

\end{itemize}

Para Mudde, cuando se habla de nacionalismo no hay diferencia en el motivo del mismo. Este puede ser étnico o político y, de forma más usual, una mezcla de ambos. Por ello, el nacionalismo que interpreta Mudde es diferente al simple amor por la patria o patriotismo. Con esto no basta, pues, para hacer una diferencia entre los que podrían llamarse nacionalistas moderados y los nacionalistas radicales. Es necesario, entonces, definir un tipo específico de nacionalismo: el nativismo. Este es básicamente la unión de nacionalismo con xenofobia. El nativismo es una \textit{definición mínima} que caracteriza a la familia política de interés. En este sentido, para este autor, la verdadera palabra del año debió haber sido nativismo \parencite{MuddeCambridge17}.\\

Sin embargo, se requieren dos elementos adicionales para llegar a una \textit{definición máxima}. Los discursos de seguridad, ley y orden que imperan en esta familia política se incluyen dentro del término autoritarismo. Por su parte, el populismo es definido por Mudde como un componente ideológico y no solamente como un recurso retórico o un estilo político. Un populista venera el ``sentido común'' del pueblo y nada es más importante que esto, ni siquiera los derechos humanos o las garantías constitucionales.\\

Para Mudde, entonces, los partidos de derecha radical de corte populista son aquellos que cuentan con tres componentes ideológicos: nativismo, autoritarismo y populismo. El orden de los términos es importante pues el nativismo es la definición mínima, al agregársele el autoritarismo se obtendría la derecha radical y, entonces, el populismo debe fungir como componente adicional a dicho término principal.\footnote{De hecho, para el autor, el populismo siempre es una especie de ideología huésped de otra ideología fundamental \parencite{MuddeRovira17}; en este caso la nativista y autoritaria.}\\ 

No obstante, aquí los llamo NAP, aunque no para seguir contribuyendo a la falta de término común. Más bien me parece que es importante destacar siempre los conceptos básicos que fueron definidos sin obscurecerlos detrás de otros. No me parece necesario, para efectos de este trabajo, construir más términos sobre los ya definidos, al tiempo que es más transparente hacer mención constante al nativismo, autoritarismo y populismo que caracteriza a la familia política.\footnote{Más aún, la argumentación por la cual Mudde llega a la etiqueta de derecha radical de corte populista reconoce que los términos \textit{derecha} y \textit{radical} deben ser interpretados de manera muy particular. El término radical, por ejemplo, se refiere a una oposición fundamental a los valores de la democracia liberal. Desde mi punto de vista, ésta es una definición problemática pues el término es usado en política con un significado distinto. Ejemplifico citando dos partidos políticos que enarbolan la etiqueta radical y que no compartirían la visión de Mudde: la Unión Cívica Radical en Argentina y el Parti Radical de Gauche en Francia. Otra objeción más quisquillosa al uso del término escogido por Mudde es que me parece que las definiciones dadas para derecha y radical no se deducen lógicamente de las definiciones de nativismo y autoritarismo, lo que contradice el objetivo inicial de tener un mayor rigor teórico y contar con definiciones que no estiren conceptos, como diría Sartori. No obstante, reconozco la practicidad que el término conlleva pues estos partidos son generalmente posicionados a la derecha de la derecha tradicional--- sin importar la definición de derecha que se tenga---, como buscan \textcite{Mammone12} o \textcite{Hainsworth16a} al abogar por el término derecha extrema.} Es bajo esta delimitación de la familia política de los NAP que habré de analizar al Front National en Francia. Por ello, lo primero que debe hacerse es verificar que, efectivamente, el FN satisfaga las definiciones de un partido NAP.\\ 

\section{El Front National como movimiento NAP}

El nativismo del Front National ha estado presente desde sus inicios. Por ejemplo, ya desde 1973 se observan tintes nativistas en sus diagnósticos que identificaban ``la constitución de verdaderos barrios o ciudades extranjeras en Francia, elementos de fragmentación y que ponen en duda la unidad y la solidaridad de nuestro pueblo'' y que exigían ``poner fin a las políticas absurdas que toleran una inmigración salvaje, en condiciones materiales y morales desastrosas para los interesados y deshonrosas para nuestro país'' \parencite[traducción propia]{LeMonde12}. Otro ejemplo lo encontramos en un recorte de periódico sobre las elecciones legislativas de ese mismo año, en el que se lee ``Contra la invasión a Francia por los indeseables'' estableciendo que-- a pesar de rechazar la idea de que los franceses sean xenófobos o racistas-- la posición del FN es que ``no es tolerable que nuestro país se haya convertido en un basurero abierto a los buenos para nada, a los defectuosos, a los delincuentes, a los criminales''\parencite[traducciones propias]{LeTemps17}.\\

Otro punto donde se refleja claramente el nativismo del FN es en sus frases o \textit{slogans} de campaña. En 1978, presenta un cartel con la frase ``Un millón de desempleados, son un millón de inmigrantes de más. Los franceses primero.''\footnote{\textit{Un million de chômeurs, c'est un million d'immigrés en trop. Les Français d'abord."} \parencite{LeMonde12}.}. Después, en los años 80 surge el término \textit{preferencia nacional} \parencite{LeTemps17}. Este término incluye al chauvinismo de bienestar en el que la ayuda social está reservada a los miembros del grupo nativo. Esta posición continuó reflejada en el programa político del FN en 2007, pero reforzada, pues se propuso también incluirlo como un principio en el preámbulo de la constitución \parencite{LObs07}. El término evolucionó después a \textit{prioridad nacional} como puede leerse en la página  6 de la plataforma presidencial en 2012 \parencite{LePen12}. Finalmente, tanto en 2007, 2012 como en 2017 las plataformas del FN propusieron eliminar la obtención de la nacionalidad francesa por derecho de suelo así como la binacionalidad, permitida únicamente para aquellos que posean otra nacionalidad europea \parencites{LObs07}{LePen12}{BBC17}.\\

Por su parte, el autoritarismo también ha estado presente en el programa político del FN. Íntimamente ligados a la inmigración dentro del discurso nativista de su líder histórico, Jean-Marie Le Pen, el crimen, la inseguridad, la ley y el orden han tenido también un lugar constante en la plataforma frontista. En 2001, a tan solo unos días de los atentados del 11 de septiembre en Nueva York, Le Pen aprovechó para señalar como amenazas, no realmente al terrorismo de Bin Laden sino todos los problemas de seguridad internos que él identificaba \parencite{ViePublique01}. Baste el siguiente extracto de su discurso para ejemplificarlo:

\begin{quote}
Los franceses enfrentan una dramática explosión de criminalidad, de violencia, de tráficos múltiples, el número de violaciones, de asesinatos y de actos de barbarie, así como el riesgo del terrorismo. La seguridad, sin embargo, es la primera de las libertades. Me comprometo, por una política de firmeza y de voluntad, basada sobre la tolerancia cero, a restaurar el orden y la ley y a organizar un referendum sobre el restablecimiento de la pena de muerte para los crímenes más graves. 
\end{quote}

La propuesta de la pena de muerte--- quizás la más viva prueba del autoritarismo--- no desaparecería pronto del discurso del FN. El ofrecimiento de restablecerla continuó en la plataforma de Jean-Marie Le Pen en las elecciones de 2007 junto con otras propuestas como disminuir la edad de responsabilidad penal de menores a los 10 años \parencite{LObs07}.\\ 

Asimismo, la sección sobre seguridad de la plataforma política de Marine Le Pen--- nueva lideresa desde 2011--- mantiene la línea dura de su padre: una política de tolerancia cero sería instaurada, los ataques organizados contra las fuerzas del orden serían fuertemente reprimidos, los efectivos policiacos habrían de aumentar, las sanciones contra reincidentes serían acrecentadas, aquella persona condenada a un año o más de prisión por reincidencia perdería todas las prestaciones sociales y, de nueva cuenta, se propondría por referendum reinstaurar la pena de muerte, así como la cadena perpetua como ``alternativas para reforzar nuestro arsenal penal'' \parencite{LePen12}. En este mismo programa podemos encontrar más evidencia de que el autoritarismo y el nativismo están íntimamente ligados dentro del FN. Por ejemplo, el ``racismo antifrancés'' como motivación de un crimen o delito sería considerado como un fuerte agravante y debería acrecentar la pena.\\

Finalmente, quedan por presentar ejemplos del populismo frontista. En este sentido Jean-Marie Le Pen siempre buscó hacer referencia a una élite corrupta constituida por lo que llamó \textit{le gang de l'établissement} \parencite{Leprince16}, de manera particular por la \textit{banda de los cuatro}, en referencia a los 4 partidos tradicionales en Francia--- 2 de derecha y 2 de izquierda--- \parencite{Boily05}.\\ 

Debido a que el populista se considera intérprete de la voluntad del pueblo, misma que es un valor en sí misma, la predilección por métodos democráticos directos es un síntoma frecuente del populismo. En Jean-Marie Le Pen lo vemos con sus propuestas sobre el referendum, no solo para la pena de muerte sino para todas las reformas fundamentales \parencite{LObs07}. Su carisma y autoproclamada vocación de darle voz al pueblo confirman ese populismo: en la elección presidencial del 2002 sus carteles rezaban \textit{Le Pen, le peuple} \parencite{Gross16}. Sin embargo, Jean-Marie Le Pen ha sido, desde mi punto de vista, ampliamente superado por su hija en términos populistas.\\ 

Marine Le Pen no solo ha mantenido el énfasis en la democracia directa y los referendums, sino que en su plataforma de 2012 propuso que el referendum fuera la \textit{única} manera de modificar la constitución \parencite[énfasis mío]{LePen12}. La centralidad de Marine como la líder populista es clara. Los logos del partido y la flama tricolor han pasado a un segundo plano detrás del rostro omnipresente de Marine Le Pen. Tembién decía yo, al explicar la terminología de Mudde, que el populista venera el sentido común del pueblo; nunca más clara esta posición en el FN como bajo el slogan \textit{la force du bon sens} de las listas \textit{Rassemblement Bleu Marine}: la fuerza del sentido común bajo el reagrupamiento azul marino, en clara referencia a la lideresa. Más aún, en los últimos años el slogan de las grandes campañas frontistas ha sido \textit{Marine, au nom du peuple!} \parencite{Gross16}. Uno solo puede conjeturar que el caracter personalista del movimiento se acentuará con el abandono del nombre Front National y su cambio por \textit{Rassemblement National} que Marine Le Pen logró en 2018.