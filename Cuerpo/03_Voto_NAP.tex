\chapter{Teorías sobre el voto NAP}

Dentro de la metodología de política comparada existen cuatro grandes paradigmas: institucional, instrumental, cultural y estructural \parencite{Uribe16}. En esta tesis, estoy interesado en explorar las \textit{configuraciones sociales} donde se desarrolla o no el voto por un partido NAP específico. Este análisis, a primera vista, parece más prometedor desde una perspectiva cultural o estructural. Más aún, si uno hace una revisión hemerográfica o en internet sobre los motivos de éxito de Trump, el Brexit o el FN encontramos que predominan fuertemente las explicaciones de carácter económico y/o cultural \parencites{Beauchamp16a}{Beauchamp16b}{Carney16}{Tesler16}{Sides16} {Arnade16}. ¿Son los desplazados por la globalización los que votan por los NAP? Los trabajadores del Rust Belt le habrían costado la presidencia a Hillary Clinton, la escolaridad es un clivaje frecuentemente mencionado, existiría un voto de izquierda económica detrás de estos movimientos... ¿O es en realidad una reacción cultural frente a la presencia de aquellos que se consideran como el \textit{Otro}: musulmanes, latinos, negros, inmigrantes en general? Estos fenómenos se están dando en sociedades con población predominantemente blanca y dentro de la cultura occidental, lo que podría interpretarse como un choque de culturas.\footnote{¿Habrá advertido Huntington que el choque se daría desde el seno mismo de las sociedades europeas y estadounidense?} Así pues, desde el estudio académico, el debate parece estar entre dos \textit{resentimientos} distintos, el económico y el racial/cultural, sin que haya \textit{a priori} uno incontroversialmente dominante sino que más bien están relacionados entre sí \parencites{Inglehart16}{Ivarsflaten14}. Entonces, ¿cómo se aproximan los paradigmas estructural y cultural a dichas explicaciones?\\  

\section{Estructuralismo y Culturalismo} 

Desde el paradigma estructural, se dice que ``las estructuras condicionan el resultado'' \parencite[69]{BV08}. El concepto tradicional de estructura se refiere a grupos con características materiales distintas. Por ejemplo, el estructuralista socioeconómico por antonomasia resulta Marx. Para él, las profundas fuerzas sociales seguirían un proceso histórico lineal que pasaría del feudalismo al capitalismo y después al socialismo, para culminar en el comunismo \parencite{BV08}. Los cambios tecnológicos impulsarían un sistema económico con nuevas clases sociales que desplazarían las viejas instituciones, refiere \textcite{Heilbroner92} al hablar de la teoría marxista. Con la revolución industrial habrían llegado dos clases sociales encontradas: la burguesía y el proletariado. La Historia marxista sería inevitable, dice el Manifiesto Comunista \parencite[148]{Heilbroner92}: 

\begin{quote}
[El desarrollo de la industria moderna] derriba los fundamentos mismos por los que la burguesía produce y apropia la producción. Lo que la burguesía produce entonces son, sobre todo, sus propios sepultureros. Su caída y la victoria del proletariado son igualmente inevitables...
\end{quote}

Hasta ahora no hemos verificado empíricamente esas afirmaciones de Marx y Engels, pero varias de sus herramientas teóricas siguen vigentes. Lo que la lucha de clases entre el proletariado y la burguesía ejemplifica es que, dentro de las sociedades, existen grupos cuyos intereses están enfrentados. La pertenencia a una de estas clases determinaría, primordialmente, las preferencias políticas de un individuo.\\

Por otro lado, el enfoque cultural tiene como unidad de análisis las ideas y valores existentes en una sociedad. 
De acuerdo con Sheri \textcite{Berman01}, una de las preguntas teóricas que este paradigma busca responder es cómo las variables culturales e ideológicas influyen el comportamiento político; en otras palabras, ¿cuáles son los mecanismos causales que conectan dichas variables con resultados políticos específicos? Un ejemplo de esto es el libro \citetitle{AlmondVerba63} \parencite{AlmondVerba63}--- generalmente considerado como el precursor de esta corriente--- que propone cómo diferentes culturas políticas llevarían a distintos grados de estabilidad democrática. Otro ejemplo son los análisis que asocian las preferencias partidarias con ciertos valores e ideas, como el que realiza Ronald Inglehart en 1977 con su teoría de la revolución silenciosa. Para Inglehart, las condiciones económicas de la posguerra con la que crecieron los jóvenes europeos--- mejores con relación a las de sus padres--- habrían hecho que se formaran en valores postmateriales como el sentido de pertenencia y la realización personal \parencite{Kesselman79}. Esto, a su vez, se relacionaría con preferencias por partidos que él llama libertarios de izquierda.\footnote{El término en inglés es \textit{left-libertarian}, ver \textcite{Inglehart16}.}\\ 

Considero, siguiendo en parte a \textcite{Sewell92}, que estos dos paradigmas pueden coexistir. Las estructuras pueden ser culturales, así como cambios en las condiciones materiales pueden llevar a cambios ideológicos. En el argumento de Inglehart esto es claro, pero la implicación va más allá. Los desarrollos económicos, así como los cambios en las relaciones de poder o la entrada de nuevos grupos sociales puede forzar un replanteamiento de las creencias existentes \parencite[235]{Berman01}. Cuando hablo de estructuras, entonces, me refiero en un sentido informal a las condiciones ideológicas o materiales que moldean diferentes grupos sociales con intereses distintos. Así pues, en lo subsecuente es bajo esta idea general de estructuras que planteo las posibles explicaciones del voto NAP. A continuación exploro algunas teorías sobre cómo las estructuras sociales y los grupos que éstas forman, guían las preferencias electorales de los votantes en relación con los movimientos NAP.

\section{Clivajes y Escolaridad}

Un constructo teórico fundamental para mi propósito es el de clivaje político, pues el principal foco de la teoría de clivajes es la ``evolución a largo plazo de la estructura social'', en donde yacen las ``fuerzas más fundamentales de la política'' \parencite[4]{Bornschier09}. Rokkan y Lipset son las referencias obligadas con respecto a los clivajes tradicionales, pero Bartolini presenta una conceptualización formal clara. Un clivaje es una división política que cuenta con tres elementos: división socioestructural, identidad colectiva y manifestación organizacional.\footnote{Citados en \textcite{Bornschier09}.} Es decir, para que una división en la sociedad pueda considerarse como un clivaje, debe existir una diferencia clara en términos socioestructurales, y que los miembros de cada grupo que constituya dicha distinción formen una identidad colectiva y cierta capacidad de movilización. Estas tres características, la teoría indica, llevarían a que los miembros de cada grupo del clivaje voten de acuerdo a los intereses del mismo. La pertenencia a uno de los lados de la estructura determinaría las preferencias políticas del individuo, traducidas en su voto.\\

Con relación a los movimientos NAP, un clivaje determinante resultaría ser el de ciudadanos escolarizados frente a los no escolarizados.\footnote{Muchos hablan de ciudadanos educados y no educados, pero considero que el término correcto debe ser escolarizados o academizados pues la diferencia socioestructural es la asistencia a una institución académica de nivel medio superior o superior, cosa que no garantiza \textit{per se} la \textit{educación} del individuo, máxime que este es un concepto sujeto a debate, súmamente subjetivo, esquivo y cargado.} La primera realización de la segunda vuelta presidencial en Austria en 2016 arrojó una clara división entre aquellos con nivel escolar obligatorio y aquellos con niveles de escolaridad media superior y superior.\footnote{Ver los resultados de encuestas referenciados por \textcite{Hoare16}.} En Estados Unidos, de acuerdo con Nate \textcite{Silver16}, la escolaridad fue el clivaje que mejor predijo quién votaba por Donald Trump. \textcite{Rae16} encontró también una fuerte relación entre el porcentaje de voto antieuropeo en el referéndum británico de 2016 y el porcentaje a nivel local de personas poco escolarizadas. Este clivaje, de acuerdo con Hervé \textcite[cap. VII \textit{Le pari}]{LeBras15}, tiene consecuencias socioeconómicas directas al tiempo que contribuye a la formación de una clase--- al menos en Francia--- temerosa frente a su posición en la estructura social y susceptible de votar por el Front National. 

\section{La Clase Desplazada por la Globalización}

Por otro lado, en la discusión actual sobre los movimientos NAP, no es raro encontrar referencias al fascismo.\footnote{Por ejemplo, el número de octubre de 2016 de la revista Letras Libres, dedicado a Donald Trump, presentó una sugerente portada con la foto del magnate y un bigote al estilo Hitler formado por las palabras \textit{Fascista Americano} \parencite{Fascista16}. Otro ejemplo es el carácter neofascista de Norbert Hofer, ver \textcite{Hoare16}.} En este sentido resulta pertinente recordar el estudio que desarrolló Barrington Moore respecto del fascismo del siglo XX.\\ 

Desde una perspectiva estructural, Moore explica el surgimiento del fascismo en Alemania, Italia y Japón. La modernización conservadora y la industrialización, frutos de una revolución desde arriba, llevaron a contradicciones estructurales fuertes en estos países. El fascismo fue un intento de hacer \textit{popular y plebeyo} el conservadurismo y el reaccionismo, que habían perdido su legitimidad, dice \textcite[447]{Moore66}. La característica principal del fascismo del siglo XX es el anticapitalismo plebeyo. En términos de voto, por ejemplo, los Nazis fueron más populares entre aquellos que tenían menos y estaban más desfavorecidos \textit{con relación al área particular en la que vivían} \parencite[448-449]{Moore66}. \textcite{Sides16} presentan cierta evidencia de que este fue el caso en Estados Unidos con el apoyo hacia Donald Trump, \textcite{Arnade16} habla de la concentración de la prosperidad en el Reino Unido como posible catalizador del voto para el Brexit y \textcite{LeBras15} examina el mapa del voto frontista en Francia frente a un índice de desigualdad socioeconómica en el mismo espíritu con el que Moore refiere el voto Nazi en el espacio rural alemán de inicios de los años treinta pero de una manera mucho más detallada.\\

Ciertamente Moore no es el único teórico que señala el vínculo entre una fuerte modernización económica y el voto hacia los movimientos extremos por parte de clases desfavorecidas: 
\begin{quote}
Al evaluar su propia situación las personas hacen comparaciones con la situación de los demás. Los votantes con una situación socioeconómica débil pueden evaluar esta posición de forma más negativa en tanto vivan en una región acaudalada. Entonces, desigualdades en el ingreso pueden traducirse en apoyo a partidos de derecha extrema \parencite[144, traducción propia]{Coffe07}.
\end{quote} Esta línea de investigación argumentaría por qué estamos viviendo hoy estos movimientos NAP. Para Branko \textcite{Milanovic16}, la globalización iniciada a finales del siglo XX--- y vigente en nuestro siglo--- constituye el más grande cambio en la distribución de los ingresos desde la revolución industrial. Esta redistribución ha traído crecimiento económico, pero de manera muy variada a lo largo de los percentiles poblacionales. Así como el 5\% más rico ha seguido creciendo por encima del promedio, los primeros 70 percentiles lo han hecho a diferentes tasas. Sin embargo, el crecimiento que llama más la atención es el de la población entre el percentil 70 y 95: han crecido menos que la media o incluso han visto su ingreso real disminuir. ¿Quiénes son estas personas? Siete de cada 10 provienen de los \textit{viejos países ricos} de la OCDE.\footnote{Esta evidencia ha llegado a ser conocida como la gráfica del elefante debido a su forma. Ver \textcite{Milanovic16}.} Precisamente muchos de los países en los cuales vemos estos movimientos NAP prosperar.\footnote{Claro que deducir que esto es prueba de que son estas personas quienes votan por los movimientos NAP sería un error, pues se caería en un caso de falacia ecológica.}\\

\section{Teorías del Conflicto e Interés económico}

Otros modelos que pudieran encajar con este comportamiento de crecimiento en las preferencias por movimientos NAP son la \textit{teoría del conflicto} y la \textit{teoría del interés económico}. La primera constituye una explicación consistente con ciertos argumentos culturales sobre la motivación racista.\footnote{Esta teoría es expuesta por autores como \textcite{Blalock67,Olzak92}--- citados por \textcite{Coffe07}---  o \textcite{Petersen02}, referido por \textcite{Beauchamp16b}.} La violencia étnica, expresada en manifestaciones xenófobas propias de los movimientos NAP, proviene del resentimiento. La clase privilegiada de la sociedad--- pensemos aquí en el tradicional grupo de hombres blancos que mencionan los culturalistas--- comienza a acumular un ``sentimiento de injusticia cuando ve su privilegio escurrirse hacia las manos de otro grupo que no lo tenía antes'', por lo que una ``causa de la violencia étnica es el cambio en estatus legal y político de los grupos étnicos mayoritarios y minoritarios''.\footnote{\textcite{Petersen02} citado por \textcite{Beauchamp16b}, traducción propia.} Este miedo a ``ir cayendo'' en la estructura social es lo que empuja a muchos franceses a \textit{apostar} por el Front National con su voto, de acuerdo a \textcite{LeBras15}. Así, \textcite{Blalock67} y \textcite{Olzak92} teorizan sobre que el conflicto aumenta en zonas de problemas económicos, como el desempleo, y ahí se suceden reacciones exclusionistas por parte de los grupos mayoritarios;{\color{Red} aunque aquí podrían insertarse referencias que señalen que este resentimiento no puede ser puramente económico y que las reacciones culturales se dan incluso en zonas económicamente beneficiadas. La teoría del conflicto, pues, no se reduce a la pérdida de privilegios económicos sino importantemente, socioculturales.}. Si al resentimiento sumáramos el miedo por competir por la escasez de los recursos, podría sugerirse el por qué los trabajadores manuales con menores niveles escolares tenderían a reaccionar frente a los grupos de inmigrantes. Esta es la lógica de la teoría del interés económico: la presencia de inmigrantes resulta en un aumento en la competencia por recursos escasos y, por lo tanto, en conflictos entre grupos sociales \parencite[144]{Coffe07}. Estas dos teorías son consistentes con el concepto de chauvinismo del bienestar: la propuesta de que los beneficios que aporta el estado de bienestar deben estar reservados exclusivamente para el grupo nativo, no para los grupos de fuera.\footnote{\textcite{Kitschelt95} citado por \textcite{Coffe07}. Observar también el paralelismo con el marco teórico estructural \textit{à la} Tilly sobre grupos miembro y grupos retadores \parencite{Skocpol79}.} Esto encaja con el nativismo que identifica Mudde en los grupos NAP \parencite{Mudde07a,Beauchamp16a} y con las propuestas concretas de sus líderes.\footnote{Ver por ejemplo el reclamo de Marine Le Pen para terminar con la educación gratuita a hijos de inmigrantes \parencite{LePen16}.}\\

Aunque probablemente menos verificable empíricamente, otra de las líneas teóricas estructurales sobre el voto por los partidos NAP lo encontramos en las referencias de \textcite{Valentino13}. Aunado a las explicaciones que ya he mencionado--- mismas que ellos agrupan dentro de la categoría de la \textit{hipótesis de la competencia del mercado laboral}--- encontramos las \textit{hipótesis fiscales}. Esta hipótesis puede contribuir a explicar la oposición de la, así llamada, pequeña burguesía hacia los inmigrantes. Éstos supondrían una presión adicional a la seguridad social y, consigo, la necesidad de incrementar los impuestos, un aumento en los costos de la educación y más cargas a la infraestructura \parencite[150]{Valentino13}. La preferencia de clase de la pequeña burguesía, que tiene negocios propios y busca que sus hijos tengan grados universitarios, es altamente contraria al aumento de impuestos y costos. Si los inmigrantes son vistos como chivos expiatorios para estos riesgos, se entendería por qué hay un número elevado de votos por los NAP entre esta clase social.\\

Estos capítulos han sido un marco teórico sobre el problema de los movimientos NAP en general y el Front National en particular. La siguiente parte de la tesis, por otro lado, servirá de marco teórico sobre el paradigma estadístico--- y sus herrramientas--- bajo el que realizaré el análisis.  