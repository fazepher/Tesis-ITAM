\chapter{Cómputo bayesiano}

Cuando alguien toma un primer curso de probabilidad o estadística, uno de los primeros ejemplos de variables aleatorias que le es presentado seguramente es el modelo de ensayos Bernoulli. Y en más de una ocasión la forma de ilustrar o enseñar el modelo es lanzando volados \textit{simulando} la realización de una variable aleatoria Bernoulli. La idea es que cada lanzamiento es independiente y se podría llegar a tener una muestra lo suficientemente grande como para empezar a mostrarle al alumno cómo se satisfacen ciertas propiedades teóricas del modelo como su media o su varianza. Gracias al avance tecnológico, hoy podemos ya no tenemos que lanzar volados físicamente sino que los simulamos desde una computadora. A partir de la generación de números aleatorios,\footnote{Técnicamente pseudo aleatorios debido a la capacidad finita de las computadoras pero diseñados de manera tal que satisfagan las propiedades básicas de la generación de números aleatorios {\color{Red} citar Simulation Ross quizás}.} también es posible generar observaciones de otras distribuciones además de un volado. Así, hay métodos básicos para simular observaciones de la distribución Binomial, Poisson o Normal, por ejemplo.\\

La utilidad de los métodos de simulación--- además de la tautología de imitar el fenómeno de estudio--- está en que, en teoría, podemos generar tantas observaciones como necesitemos para representar de buena manera las distribuciones que estemos estudiando.\\ 

\begin{itemize}
\item Queremos una muestra suficientemente grande de la posterior. Para ello requerimos un método que genere una realización \textit{independiente} de la posterior. 

\item Una distribución límite de una Cadena de Markov es tal que no cambia. Esto quiere decir que si una simulación de una cadena de Markov llega a la distribución límite, cada nueva simulación se genera de manera independiente. 

\item Si no conozco completamente una distribución posterior, ¿existe alguna cadena de Markov cuya distribución límite sea esta distribución posterior? Si existiera entonces podría simular la cadena y, eventualmente, llegar a la distribución límite y, entonces, ya podría simular una muestra aleatoria de mi distribución posterior.

\item Una muestra suficientemente grande de esta distribución posterior satisface la única receta. Los resúmenes de Monte Carlo permiten la inferencia axiomática, por llamarla de una manera.  
\end{itemize}




