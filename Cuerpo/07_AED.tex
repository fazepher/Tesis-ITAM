\chapter{Datos franceses}

Como adelantaba en la introducción, la tercera parte de este trabajo consiste concretamente en el modelado estadístico de los datos franceses que permitan la exploración de las configuraciones sociales que favorecieron o inhibieron el voto por el \textit{Front National} en las elecciones Presidenciales y Legislativas de 2007 y 2012. Por lo mismo, debo comenzar  presentando dichos datos y realizando un análisis exploratorio.

\section{Datos electorales}

En las elecciones de 2007 la derecha fue la ganadora, como puede verse en los \textbf{Cuadros \ref{tbl:Resul_Oficiales_P07} y \ref{tbl:Resul_Oficiales_L07}}. En primer lugar, Nicolas Sarkozy obtuvo la presidencia con el 53.06\% de los votos en la segunda vuelta, frente a la candidata socialista Ségolène Royal. En las elecciones legislativas, ya con Sarkozy como presidente electo, su partido, el UMP, obtuvo 313 escaños en la Asamblea Nacional; además 22 diputados electos compitieron bajo la etiqueta de Mayoría Presidencial. Por su parte, Jean-Marie Le Pen como candidato presidencial frontista en 2007 obtuvo 10.44\% de la votación efectiva en la primera vuelta, quedando en cuarta posición y fuera de la segunda vuelta. Asimismo, el FN no logró conseguir ningún diputado a pesar de obtener más de 1 millón de votos en las primeras vueltas.\\ 

\begin{table}[h]
\centering
\resizebox{\linewidth}{!}{
\begin{tabular}{l c r r r r}
\multicolumn{6}{c}{\textbf{Elecciones Presidenciales 2007}} \\[5pt] 
\multirow{2}{*}{\textbf{Candidatura}} & 
\multirow{2}{*}{\textbf{Partido}} & 
\multicolumn{1}{c}{\textbf{Votos}} & \textbf{\% Ef.} & 
\multicolumn{1}{c}{\textbf{Votos}} & \textbf{\% Ef.}\\ 
& &  
\multicolumn{2}{c}{1ra vuelta} & 
\multicolumn{2}{c}{2da vuelta} \\[2pt] 
\hline
 & & & & & \\[\dimexpr-\normalbaselineskip+3pt]
Nicolas Sarkozy & UMP & 11,448,663 & 31.18 & 18,983,138 & 53.06 \\
Ségolène Royal & PS & 9,500,112 & 25.87 & 16,790,440 & 46.94\\
\hdashline
François Bayrou & UDF & 6,820,119 & 18.57 & & \\
Jean Marie Le Pen & FN & 3,834,530 & 10.44 & & \\
Olivier Besancenot & LRO & 1,498,581 & 4.08 & & \\
Philippe de Villiers & MPF & 818,407 & 2.23 & & \\
Marie-George Buffet & PC & 707,268 & 1.93 & & \\
Dominique Voynet & Verts & 576,666 & 1.57 & & \\
Arlette Laguiller & LO & 487,857 & 1.33 & & \\
José Bové & Indep. & 483,008 & 1.32 & & \\
Frédérick Nihous & CPNT & 420,645 & 1.15 & & \\
Gérard Schivardi & PT & 123,540 & 0.34 & & \\[3pt]
\hline
 & & & & & \\[\dimexpr-\normalbaselineskip+3pt]
\multicolumn{2}{l}{\textbf{Votación efectiva}} 
& 36,719,396 &
& 35,773,578 & \\
\multicolumn{2}{l}{Blancos o nulos} 
& 534,846 & 
& 1,568,426 & \\[3pt]
\hline
 & & & & & \\[\dimexpr-\normalbaselineskip+3pt]
\multicolumn{2}{l}{\textbf{Votación emitida}} 
& 37,254,242 & 
& 37,342,004 & \\
\multicolumn{2}{l}{Abstenciones} 
& 7,218,592 & 
& 7,130,729 & \\[3pt]
\hline
 & & & & & \\[\dimexpr-\normalbaselineskip+3pt]
\multicolumn{2}{l}{\textbf{Lista nominal}} 
& 44,472,834 & 
& 44,472,733 & \\
\end{tabular}
}
\caption{Resultados de las elecciones presidenciales francesas de 2007, resultando presidente electo Nicolas Sarkozy (UMP). Fuente: elaboración propia con los datos oficiales del Ministerio del Interior.}
\label{tbl:Resul_Oficiales_P07}
\end{table}

\begin{sidewaystable}[ph!]
\centering
\resizebox{\linewidth}{!}{
\begin{tabular}{l c r r c r r c c}
\multicolumn{9}{c}{\textbf{Elecciones Legislativas 2007}} \\[5pt] 
\multicolumn{2}{c}{\multirow{2}{*}{\textbf{Plataforma política}}} & 
\textbf{Votos} & 
\textbf{\% Ef.} & 
\textbf{Asientos} &
\textbf{Votos} & 
\textbf{\% Ef.} & 
\textbf{Asientos} &
\multirow{2}{*}{\textbf{Total de Asientos}}\\ 
\multicolumn{2}{c}{} 
& \multicolumn{3}{c}{1ra vuelta} 
& \multicolumn{3}{c}{2da vuelta} 
& \\[2pt] 
\hline
 & & & & & & & &\\[\dimexpr-\normalbaselineskip+3pt]
Union pour un Mouvement Populaire & UMP 
& 10,289,737 & 39.54 & 98 & 9,460,710 & 46.36 & 215 & 313 \\
Socialiste & SOC 
& 6,436,520 & 24.73 & 1 & 8,624,861 & 42.27 & 185 & 186 \\
Majorité Présidentielle & MAJ
& 616,440 & 2.37 & 8 & 433,057 & 2.12 & 14 & 22\\
Parti Communiste Français & COM 
& 1,115,663 & 4.29 & - & 464,739 & 2.28 & 15 & 15\\
Divers Gauche & DVG
& 513,407 & 1.97 & - & 503,556 & 2.47 & 15 & 15\\
Divers Droite & DVD 
& 641,842 & 2.47 & 2 & 238,588 & 1.17 & 7 & 9\\
Radicales de Gauche & RDG
& 343,565 & 1.32 & - & 333,194 & 1.63 & 7 & 7\\
Les Verts & VEC
& 845,977 & 3.25 & - & 90,975 & 0.45 & 4 & 4\\
Union pour la Démocratie Française & UDFD 
& 1,981,107 & 7.61 & - & 100,115 & 0.49 & 3 & 3\\
Mouvement pour la France & MPF
& 312,581 & 1.20 & 1 & - & - & - & 1\\
Divers & DIV 
& 267,760 & 1.03 & - & 33,068 & 0.16 & 1 & 1\\
Régionaliste & REG
& 133,473 & 0.51 & - & 106,484 & 0.52 & 1 & 1\\
Front National & FN 
& 1,116,136 & 4.29 & - & 17,107 & 0.08 & - & -\\
Extrême Gauche & EXG 
& 888,250 & 3.41 & - & -  & - & - & -\\
Chase Pêche Nature Traditions & CPNT 
& 213,427 & 0.82 & - & - & - & - & -\\
Ecologistes & ECO 
& 208,456 & 0.80 & - & - & - & - & -\\
Extrême Droite & EXD 
& 102,124 & 0.39 & - & - & - & - & -\\[3pt]
\hline
 & & & & & & & & \\[\dimexpr-\normalbaselineskip+3pt]
\multicolumn{2}{l}{\textbf{Votación efectiva}} 
& 26,026,465 & &
& 20,406,454 & & & \\
\multicolumn{2}{l}{Blancos o nulos} 
& 495,357 & &
& 722,585 & & & \\[3pt]
\hline
 & & & & & & & &\\[\dimexpr-\normalbaselineskip+3pt]
\multicolumn{2}{l}{\textbf{Votación emitida}} 
& 26,521,822 & &
& 21,129,039 & & & \\
\multicolumn{2}{l}{Abstenciones} 
& 17,374,011 & &
& 14,096,209 & & & \\[3pt]
\hline
 & & & & & & &\\[\dimexpr-\normalbaselineskip+3pt]
\multicolumn{2}{l}{\textbf{Lista nominal}} 
& 43,895,833 & &
& 35,225,248 & & & \\
\end{tabular}
}
\caption{Resultados de las elecciones legislativas de 2007 para el conjunto del territorio francés, incluyendo el ultramar. Fuente: elaboración propia con los datos oficiales del Ministerio del Interior.}
\label{tbl:Resul_Oficiales_L07}
\end{sidewaystable}

En 2012, sin embargo, Sarkozy perdió la reelección frente al socialista François Hollande, como puede verse en el \textbf{Cuadro \ref{tbl:Resul_Oficiales_P12}}. En el poder legislativo, la izquierda tomó el control del Hemiciclo gracias a los 280 diputados electos bajo las siglas del PS (\textbf{Cuadro \ref{tbl:Resul_Oficiales_L12}}). En lo que respecta al FN, este tuvo un crecimiento considerable. Con Marine Le Pen como lideresa y candidata, el partido obtuvo 17.90\% de la votación efectiva en la primera vuelta presidencial. Dicho porcentaje representó el 3er lugar en la elección, mejorando el 4to puesto de 2007. No obstante, de nueva cuenta fue insuficiente para disputar la segunda vuelta electoral. En las elecciones legislativas, sin embargo, el FN logró alrededor de 3 millones y medio de sufragios en la primera vuelta--- equivalentes al 13.60\%--- y consiguió 2 diputaciones.\\

\begin{table}[h]
\centering
\resizebox{\linewidth}{!}{
\begin{tabular}{l c r r r r}
\multicolumn{6}{c}{\textbf{Elecciones Presidenciales 2012}} \\[5pt] 
\multirow{2}{*}{\textbf{Candidato(a)}} & 
\multirow{2}{*}{\textbf{Partido}} & 
\multicolumn{1}{c}{\textbf{Votos}} & \textbf{\% Ef.} & 
\multicolumn{1}{c}{\textbf{Votos}} & \textbf{\% Ef.}\\ 
& &  
\multicolumn{2}{c}{1ra vuelta} & 
\multicolumn{2}{c}{2da vuelta} \\[2pt] 
\hline
 & & & & & \\[\dimexpr-\normalbaselineskip+3pt]
François Hollande & PS & 10,272,705 & 28.63 & 18,000,668 & 51.64\\
Nicolas Sarkozy & UMP & 9,753,629 & 27.18 & 16,860,685 & 48.36 \\
\hdashline
Marine Le Pen & FN & 6,421,426 & 17.90 & & \\
Jean-Luc Mélenchon & FG & 3,984,822 & 11.10 & & \\
François Bayrou & MoDem & 3,275,122 & 9.13 & & \\
Eva Joly & EELV & 828,345 & 2.31 & & \\
Nicolas Dupont-Aignan & DLR & 643,907 & 1.79 & & \\
Philippe Poutou & NPA & 411,160 & 1.15 & & \\
Nathalie Arthaud & LO & 202,548 & 0.56 & & \\
Jacques Cheminade & SP & 89,545 & 0.25 & & \\[3pt]
\hline
 & & & & & \\[\dimexpr-\normalbaselineskip+3pt]
\multicolumn{2}{l}{\textbf{Votación efectiva}} 
& 35,883,209 &
& 34,861,353 & \\
\multicolumn{2}{l}{Blancos o nulos} 
& 701,190 & 
& 2,154,956 & \\[3pt]
\hline
 & & & & & \\[\dimexpr-\normalbaselineskip+3pt]
\multicolumn{2}{l}{\textbf{Votación emitida}} 
& 36,584,399 & 
& 37,016,309 & \\
\multicolumn{2}{l}{Abstenciones} 
& 9,444,143 & 
& 9,049,998 & \\[3pt]
\hline
 & & & & & \\[\dimexpr-\normalbaselineskip+3pt]
\multicolumn{2}{l}{\textbf{Lista nominal}} 
& 46,028,542 & 
& 46,066,307 & \\
\end{tabular}
}
\caption{Resultados de las elecciones presidenciales francesas de 2012, resultando presidente electo François Hollande (PS). Fuente: elaboración propia con los datos oficiales del Ministerio del Interior.}
\label{tbl:Resul_Oficiales_P12}
\end{table}

\begin{sidewaystable}[ph!]
\centering
\resizebox{\linewidth}{!}{
\begin{tabular}{l c r r c r r c c}
\multicolumn{9}{c}{\textbf{Elecciones Legislativas 2012}} \\[5pt] 
\multicolumn{2}{c}{\multirow{2}{*}{\textbf{Plataforma política}}} & 
\textbf{Votos} & 
\textbf{\% Ef.} & 
\textbf{Asientos} &
\textbf{Votos} & 
\textbf{\% Ef.} & 
\textbf{Asientos} &
\multirow{2}{*}{\textbf{Total de Asientos}}\\ 
\multicolumn{2}{c}{} 
& \multicolumn{3}{c}{1ra vuelta} 
& \multicolumn{3}{c}{2da vuelta} 
& \\[2pt] 
\hline
 & & & & & & & &\\[\dimexpr-\normalbaselineskip+3pt]
Socialiste & SOC 
& 7,618,326 & 29.35 & 22 & 9,420,889 & 40.91 & 258 & 280 \\
Union pour un Mouvement Populaire & UMP 
& 7,037,268 & 27.12 & 9 & 8,740,628 & 37.95 & 185 & 194 \\
Divers Gauche & DVG
& 881,555 & 3.40 & 1 & 709,395 & 3.08 & 21 & 22\\
Europe-Écologie-Les Verts & VEC
& 1,418,264 & 5.46 & 1 & 829,036 & 3.60 & 16 & 17\\
Divers Droite & DVD 
& 910,034 & 3.51 & 1 & 417,940 & 1.81 & 14 & 15\\
Nouveau Centre & NCE
& 569,897 & 2.20 & 1 & 568,319 & 2.47 & 11 & 12\\
Radicales de Gauche & RDG
& 428,898 & 1.65 & 1 & 538,331 & 2.34 & 11 & 12\\
Front de Gauche & FG 
& 1,793,192 & 6.91 & - & 249,498 & 1.08 & 10 & 10\\
Parti Radical & PRV
& 321,124 & 1.24 & - & 311,199 & 1.35 & 6 & 6\\
Front National & FN 
& 3,528,663 & 13.60 & - & 842,695 & 3.66 & 2 & 2\\
Le Centre pour la France & CEN
& 458,098 & 1.77 & 1 & 113,196 & 0.49 & 2 & 2\\
Alliance Centriste & ALLI 
& 156,026 & 0.60 & - & 123,132 & 0.53 & 2 & 2\\
Régionaliste & REG
& 145,809 & 0.56 & - & 135,312 & 0.59 & 2 & 2\\
Extrême Droite & EXD 
& 49,499 & 0.19 & - & 29,738 & 0.13 & 1 & 1\\
Extrême Gauche & EXG 
& 253,386 & 0.98 & - & - & - & - & -\\
Ecologistes & ECO 
& 249,068 & 0.96 & - & - & - & - & -\\
Autres & AUT 
& 133,752 & 0.52 & - & - & - & - & -\\[3pt]
\hline
 & & & & & & & & \\[\dimexpr-\normalbaselineskip+3pt]
\multicolumn{2}{l}{\textbf{Votación efectiva}} 
& 25,952,859 & &
& 23,029,308 & & & \\
\multicolumn{2}{l}{Blancos o nulos} 
& 416,267 & &
& 923,178 & & & \\[3pt]
\hline
 & & & & & & & &\\[\dimexpr-\normalbaselineskip+3pt]
\multicolumn{2}{l}{\textbf{Votación emitida}} 
& 26,369,126 & &
& 23,952,486 & & & \\
\multicolumn{2}{l}{Abstenciones} 
& 19,712,978 & &
& 19,281,162 & & & \\[3pt]
\hline
 & & & & & & &\\[\dimexpr-\normalbaselineskip+3pt]
\multicolumn{2}{l}{\textbf{Lista nominal}} 
& 46,082,104 & &
& 43,233,648 & & & \\
\end{tabular}
}
\caption{Resultados de las elecciones legislativas de 2012 para el conjunto del territorio francés, incluyendo el ultramar. Fuente: elaboración propia con los datos oficiales del Ministerio del Interior.}
\label{tbl:Resul_Oficiales_L12}
\end{sidewaystable}

Ahora bien, como puede verificarse con las cuatro elecciones, en Francia existe una gran variedad de partidos políticos repartidos a lo largo del espectro político, tradicionalmente asociado con las nociones de izquierda y derecha. Adicionalmente, no es raro que los partidos cambien de nombres o formen coaliciones. Por ejemplo, en 2007 hubo 12 candidaturas presidenciales y para  2012 fueron 10. En las elecciones legislativas se presentaron 17 etiquetas políticas diferentes ambas ocasiones. Por ello, y para simplificar el análisis, conviene agrupar las diferentes opciones políticas bajo algunas etiquetas cualitativas generales.\\ 

Dependiendo de la posición--- aproximada--- de la plataforma política en el eje ideológico izquierda-derecha propondría los siguientes grupos. En primer lugar, debido al interés específico en el Front National, este es clasificado individualmente. Posteriormente, tenemos a los partidos tradicionales de derecha e izquierda--- el UMP\footnote{Este partido ya cambió de nombre y ahora se conoce como \textit{Les Republicains}. En 2007, después de que Nicolas Sarkozy ganara las elecciones, además de los candidatos del UMP hubo candidaturas bajo la etiqueta de la \textit{Mayoría presidencial} por lo que también las agrupo junto con el UMP.} y los socialistas del PS, respectivamente---. Además, existen plataformas de centro, así como otras izquierdas y derechas. Finalmente, encontramos opciones con un fuerte componente temático--- como los verdes o los regionalistas--- o con una plataforma política especial de forma tal que resulta más conveniente separarlos del eje y clasificarlos en una categoría residual. Así pues, en total existen 7 grupos generales: 

\begin{itemize}
\item FN
\item Derecha
\item Izquierda
\item Centro
\item Otras derechas
\item Otras izquierdas
\item Otros
\end{itemize}

En la \textbf{Figura \ref{fig:Partidos_07_12}} pueden observarse representaciones esquemáticas de esta simplificación del espectro político francés para 2007 y 2012.\\

\begin{figure}[h]
	\centering
	\begin{subfigure}{0.9\textwidth}
	\includegraphics[width = \textwidth]{Figs/FN_Francia/Partidos_07}
	\end{subfigure}	
	~
	\begin{subfigure}{0.9\textwidth}
	\includegraphics[width = \textwidth]{Figs/FN_Francia/Partidos_12}
	\end{subfigure}		
	\caption{Representación esquemática de los partidos, candidaturas y etiquetas políticas en las elecciones francesas de 2007 y 2012. Fuente: elaboración propia.}
	\label{fig:Partidos_07_12}	
\end{figure}

Debido a que el FN compitió primordialmente en las primeras vueltas en las 4 elecciones, solo me enfocaré en ellas y no en las segundas vueltas. Asimismo, el fenómeno electoral es marcadamente distinto en la metrópoli francesa que en los territorios de ultramar donde, naturalmente, el FN es prácticamente inexistente. Por tanto, el análisis también se circunscribe a los resultados dentro de la metrópoli francesa. 

\begin{figure}[h]
	\centering
	\begin{subfigure}{0.3\textwidth}
	\includegraphics[width = \textwidth]{Figs/AED/Geofacet_Distr_por_Reg_P12_FN}
	\end{subfigure}\\	
	~
	\begin{subfigure}{0.3\textwidth}
	\includegraphics[width = \textwidth]{Figs/AED/Geofacet_Distr_por_Reg_P12_Derecha}
	\end{subfigure}
	~
	\begin{subfigure}{0.3\textwidth}
	\includegraphics[width = \textwidth]{Figs/AED/Geofacet_Distr_por_Reg_P12_Izquierda}
	\end{subfigure}
	~
	\begin{subfigure}{0.3\textwidth}
	\includegraphics[width = \textwidth]{Figs/AED/Geofacet_Distr_por_Reg_P12_Centro}
	\end{subfigure}
	~
	\begin{subfigure}{0.3\textwidth}
	\includegraphics[width = \textwidth]{Figs/AED/Geofacet_Distr_por_Reg_P12_Otras_derechas}
	\end{subfigure}
	~
	\begin{subfigure}{0.3\textwidth}
	\includegraphics[width = \textwidth]{Figs/AED/Geofacet_Distr_por_Reg_P12_Otras_izquierdas}
	\end{subfigure}
	~
	\begin{subfigure}{0.3\textwidth}
	\includegraphics[width = \textwidth]{Figs/AED/Geofacet_Distr_por_Reg_P12_Otros}
	\end{subfigure}
	\caption{Representación esquemática de los partidos, candidaturas y etiquetas políticas en las elecciones francesas de 2007 y 2012. Fuente: elaboración propia.}
	\label{fig:Geofacet_Distr_Reg_P12}	
\end{figure}