\chapter{Conclusiones}

Esta tesis aborda uno de los fenómenos políticos internacionales que, desde mi punto de vista, resulta por de más interesante de estudiar: los movimientos nativistas y autoritarios de corte populista que comúnmente son denominados con etiquetas como derechas radicales o extremas. Un buen punto de partida para estudiar estos movimientos es considerar uno de los partidos más característicos de esta corriente política, el \textit{Front National} francés. Utilizando datos oficiales del censo francés realicé un modelo estadístico de regresión que explicara el voto obtenido por el FN en las elecciones presidenciales del 2012 a partir de las distintas configuraciones sociales que dichos datos definen. Así, este modelo arroja varias lecciones.\\

En primer lugar, debo decir que el análisis estadístico refuerza lo que una parte importante de la literatura revisada sugiere. El clivaje de escolaridad es, probablemente, el más importante o explicativo a la hora de estudiar a estos movimientos NAP. Una mayor presencia a nivel local de personas con escolaridad universitaria o superior inhibió más el voto frontista en 2012 que cualquier otra de las variables consideradas. Por el contrario, una configuración social con mayor presencia de personas sin escolaridad formal o cuyo máximo grado de estudios fue la preparatoria favoreció el voto por la candidata del FN. Me resulta interesante que la variable de escolaridad sea, en este sentido, la más significativa pues es una variable que puede incorporar los dos \textit{resentimientos} que la literatura relacionaría con el fenómeno NAP: el cultural y el económico. La escolaridad tiene consecuencias económicas pero fundamentalmente puedo influir en los valores e ideas que tienen los individuos.\\

En este sentido, podemos conjeturar que ambas explicaciones coexisten pero parecería que la influencia de la ansiedad económica o la consciencia de clase es menor que aquella de la ``cultura política'' entendida como el conjunto de valores sociales e ideología que guía las decisiones políticas de los individuos. La variable socioeconómica por excelencia en los estudios de sociología electoral francesa que rescato de la literatura es la categoría socioprofesional de las personas. A pesar de contribuir a la explicación, parece que la única verdadera clase social--- en el sentido tradicionalmente entendido--- que favoreció el voto frontista fue la clase obrera, confirmando la existencia de un \textit{gaucho-lepensime}. Una mayor presencia de obreros estuvo asociada con un mayor voto frontista aunque en una magnitud menor que los efectos de las categorías de escolaridad. Por el contrario, una configuración social a nivel comuna con mayores niveles de cuadros y profesiones intelectuales inhibió el voto FN en 2012. ¿Existirá una consciencia de clase entre ellos o podemos pensar que el efecto se debe más a su asociación con contextos sociales culturalmente discordantes con el mensaje xenófobo y autoritario del FN?\\ 

Más aún, la influencia que tiene una distinta composición de la población comunal en términos de grupos de edades sería evidencia a favor de teorías culturalistas como las de Inglehart. Las comunas con mayores niveles de jóvenes entre 18 y 24 años tuvieron menor afinidad con Marine Le Pen en la elección presidencial. Las comunas más ``avejentadas'', en el sentido de contar con mayor presencia de personas retiradas o de más de 65 años, también votaron menos por Le Pen. Probablemente esto se deba a que la primera es una generación con valores y preocupaciones distintas a las del FN y, la segunda, la más cercana a la tradición gaullista de la derecha usual. En el sentido opuesto, las comunas con mayores porcentajes de menores de edad reflejaron mayor apoyo al FN. Esto ameritaría un estudio más detallado pero una primera hipótesis podría sugerir una sociedad más rural y también menos cercana a los valores posmodernos de familias pequeñas, mismos que podríamos asociar políticamente a partidos de izquierda como el socialista o el movimiento ecologista. En este mismo espíritu podríamos interpretar la mayoría de efectos negativos asociados a comunas con mayor presencia de mujeres.\\

Ahora bien, contrario a lo que la mayoría de las teorías sobre el conflicto o la competencia fiscal sugerirían que causa la presencia de inmigrantes, este estudio encuentra más bien una relación negativa con el voto frontista. Ello puede deberse a distintos factores. Una mayor cantidad de inmigrantes puede llevar a que más personas de este grupo accedan al derecho al voto, ya sea por naturalización o por un efecto de generaciones siguientes. Estos ciudadanos (pro)inmigrantes tenderían a rechazar el mensaje xenófobo de Le Pen. Adicionalmente, podría ser que la conviviencia a nivel local con inmigrantes elimine el miedo al \textit{Otro} que podría estar alimentando los sentimientos nativistas. Bajo esta hipótesis, más que un conflicto directo entre grupos estaríamos hablando de una xenofobia indirecta y de la ansiedad cultural que asocia la inmigración con un declive de las tradiciones europeas.\\

¿Es el desempleo reflejo de una descomposición económica que favorezca el surgimiento de los movimientos NAP? Hay referencias que particularmente señalan la ansiedad económica causada por el desempleo o fragilidad laboral juvenil. Esta tesis no parece encontrar que estas hipótesis sean las más favorecidas. De hecho, el desempleo juvenil podría considerarse como la variable menos explicativa de todas las aquí consideradas. Si bien observamos que el desempleo en general puede tener un efecto en el voto, este sería más modesto que otros y no en un mismo sentido a nivel nacioanl. En algunos lugares mayor desempleo favoreció el voto frontista y en otros lo inhibió.\\

Esta última consideración me lleva a llamar la atención sobre algo que parecería ser una respuesta frecuente en ciencias sociales y que en ocasiones pasamos por alto: depende. Muchas veces querríamos contar con teorías prístinas y unificantes como la existencia de ``El votante FN'' y sin embargo hay que recordar que la realidad es más compleja. Lo que los modelos de esta tesis señalan es que no existe una única relación entre las variables sociales consideradas. En este sentido estaríamos hablando de diferentes efectos dependiendo del contexto. La posibilidad de llevar a cabo un modelado jerárquico por departamento en lugar de conformarse con un modelo nacional más sencillo nos permite reconocer de mejor manera la variabilidad y la incertidumbre que tenemos sobre estos efectos. Mientras que hay variables con efectos más o menos homogéneos a nivel nacional, existen otras que influyen de manera marcadamente opuesta dependiendo del departamento del que hablemos.

\section*{Trabajo futuro}

Esta tesis, como es de esperarse, más que aportar respuestas definitivas genera nuevas preguntas tanto dentro del análisis de los movimientos NAP como respecto al contexto estadístico. Por lo mismo, señalo algunas de las posibilidades para continuar el estudio así como algunos comentarios respecto del proceso de modelado utilizado.\\

La primera forma de continuar el estudio de los movimientos NAP en general y el FN en particular es ajustar el modelo aquí desarrollado a otras elecciones. Están disponibles los datos para replicar el estudio en la elección del 2007 y, de acuerdo al calendario de difusión de resultados del INSEE, el próximo año contaremos con los datos censales necesarios para aplicarlos a la elección del 2017. El análisis de estas dos elecciones ofrecerían un panorama más completo de las configuraciones sociales del FN pues en la primera el candidato era Le Pen padre y no había pasado la crisis económica de 2008-2009 mientras que en la segunda Marine Le Pen accedió a la segunda vuelta presidencial lo que ofrece la posibilidad de estudiar ambas vueltas electorales. Podría verificarse qué factores han tenido continuidad y cuáles de las variables, de haberlas, han cambiado su efecto. Más aún, podrían también estudiarse las elecciones europeas y legislativas. Estas últimas, sin embargo, tienen la desventaja de no contar con la presencia de candidaturas frontistas en todos los lugares y ser elecciones que podríamos considerar como secundarias por lo que las motivaciones para votar por las diferentes fuerzas políticas suelen ser distintas.\\

Otra ruta es realizar un estudio de política comparada. Hoy por hoy, parecería que la batuta europea de los movimientos NAP la ha tomado la Lega de Matteo Salvini en Italia. El Istat, equivalente al INSEE francés o a nuestro INEGI, también ofrece datos que podrían permitir un estudio comparado de las configuraciones sociales de este partido italiano con aquellas del FN. Asimismo, pueden explorarse comparaciones con otros movimientos NAP. Finalmente, se podrían considerar otras variables además de las aquí utilizadas.\\

No quisiera terminar este trabajo sin reafirmar mi convicción de que la dignidad humana es sagrada. Por ello, busqué estudiar una corriente ideológica que, desde mi particular punto de vista, atenta constantemente contra ella. No es sino intentando entender por qué, cómo y dónde surgen estos movimientos que podremos dar respuestas y mitigar las consecuencias dañinas que de ellos se deriven. Es natural tener miedo por el futuro, pero deseo firmemente que quienes por distintas circunstancias lo experimenten, puedan encontrar esperanza sin odio a otros seres humanos. Pues, parafraseando al papa Francisco en su viaje a Mozambique, ningún país tiene futuro si el motor que une, convoca y tapa las diferencias de sus ciudadanos es el odio \parencite{Francisco}.
 
 
