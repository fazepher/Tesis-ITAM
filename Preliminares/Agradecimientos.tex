\chapter*{Agradecimientos}
\addcontentsline{toc}{chapter}{Agradecimientos}
%\markboth{AGRADECIMIENTOS23}{AGRADECIMIENTOS} % encabezado 

Los logros individuales se comparten. Sé que nunca he estado solo en este camino. Por lo mismo, quiero aprovechar estas líneas para agradecer, a manera de reconocimiento, a todas aquellas personas que son partícipes de la satisfacción que para mí representa esta tesis.\\

En primer lugar, gracias infinitas a Dios. ``¡Cuántas maravillas has hecho, Señor, Dios mío! ¡Cuántos proyectos para nosotros! [...] Yo quisiera contarlos, publicarlos, pero son innumerables'' (Sal 40, 6). Gracias, porque Tú Señor me has dado ``un lote estupendo, ¡qué hermosa es mi herencia!'' (Sal 16, 6). Gracias, porque nunca te has alejado de mí; más me has amado cuando yo sí lo he hecho.\\

Gracias a mi familia. Mis papás siempre han sido ejemplo de humanidad, amor, esfuerzo y responsabilidad para mí y mis hermanos. Con ustedes cuatro quiero compartir todos mis logros. Gracias a mi tía Beatriz, porque siempre me ha animado y ayudado; no importa si son idiomas, investigaciones, viajes, discusiones, planes o juegos de mesa, siempre estás ahí.\\ 

Quiero agradecer también a mi tía Coco, a mis tíos Pedro Pablo y Claudia, por su cariño. Gracias a Calixto y Ana, a quienes admiro y respeto. Gracias a mis primos, especialmente, a Mónica, por la Roma y la poesía; a Paola, Andrea, Mariana, Andrés y Pedro Pablo por los cafés y los desayunos; a Ana y Fátima por las risas y el cariño. Gracias a mis terceros abuelos, Rosa María y Jorge. Si en México decimos mi casa es tu casa, ustedes fueron más allá y convirtieron el dicho en un hogar y una nueva parte de mi familia.\\

Gracias al Dr. Luis Enrique Nieto, por invitarme al congreso del ISBA, pero más por su guía constante y dedicada y su disposición para ayudarme a ser un mejor estadístico. Gracias al Dr. Goodliffe por ayudarme a entender mejor al Front National. Agradezco a mis sinodales de ambas carreras. Al Dr. Manuel Mendoza por sus comentarios y porque mi formación estadística no sería la misma sin sus clases; al Dr. Vives, porque siempre he tenido la puerta abierta para aprender, escuchar y, por qué no, compartir un mate; al Dr. Barrios, porque las buenas bases probabilísticas siempre se agradecen en Aguascalientes; al Dr. Sberro, por promover que siempre pensemos en la antítesis de nuestras ideas; al Dr. Rubén Hernández por sus consejos y su motivación. Gracias a todos mis profesores.\\

Gracias a Estudios y a FAL. Al profesor Villafranca y a Grace, por ayudarme a recorrer el ITAM desde el inicio y hasta ahora. A todo el equipo de la oficina, porque siempre hubo sonrisas y una que otra vinoterapia. En especial a Mike y Eri, por los libros y la música.\\

Gracias a Numérika, porque me ha permitido aplicar la estadística, pero además mejorando mi vida y disfrutando mi trabajo. Gracias Álvaro y Emilio, por la oportunidad y la enorme calidad humana. Gracias Flor, Santiago, Ame y Lu por ser los mejores compañeros de trabajo que pude pedir.\\

Gracias a Andrea, Chilaquil, Lucy, Mer, Moni, Omar, Paulina y Raúl porque el ITAM nos puso a caminar juntos y aquí seguimos, haya o no geografía de por medio.

