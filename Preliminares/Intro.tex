\chapter*{Introducción}
\addcontentsline{toc}{chapter}{Introducción}

Al cursar la Licenciatura en Actuaría elegí el área de concentración en estadística pues considero que esta es el puente perfecto entre mi inclinación cuantitativa y mi gusto por las ciencias sociales. Sin perder formalidad y rigor matemáticos, resulta ser ampliamente aplicable a problemas reales. Bien dijo John Tukey que la mejor parte de ser un estadístico es que puedes jugar en el patio trasero de todo mundo; el mío son los fenómenos políticos.\\

Además de Actuaría también cursé en el ITAM la Licenciatura en Relaciones Internacionales. Por ello, me resultó natural iniciar como tesis de ambas carreras un estudio estadístico relacionado con uno de los fenómenos políticos internacionales que más han despertado interés en los últimos años: aquellos movimientos populistas que usualmente son llamados de \textit{derechas extremas o radicales}.\\ 

Desarrollada particularmente en Europa, ni la academia ni los medios de comunicación logran un consenso sobre cómo llamar o clasificar a esta familia política. A pesar de ello, de acuerdo con Cas Mudde, es posible definir a la familia política con base en tres características básicas \parencites{Mudde07a}{Beauchamp16a}. En primer lugar, comparten el nativismo como forma xenófoba de nacionalismo. En segundo lugar, son movimientos autoritarios, que privilegian el discurso de la seguridad, la ley y el orden. Finalmente, los une el populismo como forma no solo de hacer política sino, más fundamentalmente, de ver a la sociedad: frente a las élites corruptas--- \textit{el ellos}--- el pueblo puro--- \textit{el nosotros}---. Mudde los llama movimientos de derecha radical populistas aunque aquí los llamaré NAP--- nativistas autoritarios de corte populista--- con el objetivo de resaltar sus características fundamentales.\\

De manera particular, el partido que pudiera considerarse \textit{pater familias} de esta corriente es el \textit{Front National} francés \parencite{Mudde07a}. Este fue fundado, entre otros personajes, por Jean-Marie Le Pen en 1972. Surgió en la escena política en 1984 al obtener algo más de 2 millones de sufragios en las elecciones europeas de ese año \parencite{LeBras15}. En 2002, Le Pen avanzó de manera sorpresiva a la segunda vuelta presidencial, aunque finalmente perdió frente a Jacques Chirac. En los últimos años, ya con Marine Le Pen--- hija del fundador--- como lideresa, el FN ha roto el bipartidismo en Francia \parencite{LeBras16};  en las elecciones presidenciales de 2017 ella también avanzó a la segunda vuelta presidencial. En junio de 2018 el partido cambió oficialmente de nombre a \textit{Rassemblement National} con miras a seguir creciendo en medio de la ola nativista que pareciera darse en el mundo.\\

Las preguntas respecto a los movimientos NAP son múltiples. ¿Por qué la gente vota por estas propuestas, siendo que algunas son abiertamente racistas? ¿Qué tipo de votante los ha apoyado y cuál los rechaza? ¿Dónde han tenido más votos? ¿Cuáles son los terrenos fértiles que han permitido este surgimiento y cuáles han fungido como barreras a su crecimiento? De hecho, es la familia política más estudiada en los últimos años \parencite{Mudde16}.\\ 

Siendo el referente tradicional de los movimientos NAP, muchos libros y artículos se han publicado en la búsqueda por entender al votante y la sociología política del Front National. La mayoría de ellos se aproximan a las preguntas mediante estudios cualitativos o con base en datos individuales provenientes de encuestas con representatividad nacional.\\ 

Esta tesis pretende complementar el entendimiento del fenómeno mediante un estudio de caso, aunque con una estrategia distinta. A partir de datos censales agregados--- también llamados ecológicos--- y técnicas de modelado estadístico jerárquico, buscaré describir las \textit{configuraciones sociales} que favorecieron o inhibieron el voto por la candidatura frontista de Marine Le Pen en la elección presidencial de 2012. Es decir, ¿acaso las zonas de mayor desempleo votaron en mayor medida por Le Pen?, ¿fueron más bien los lugares de mayor inmigración los que apoyaron a la candidata frontista?, ¿lugares con mayor presencia de personas escolarizadas experimentaron menores niveles de apoyo al FN?\\

Este enfoque de modelado jerárquico de datos ecológicos, tiene la ventaja, como bien apunta Joël \textcite{Gombin05}, de reconocer y modelar la variabilidad territorial de los fenómenos políticos, así como la importancia del contexto social en el que los individuos se desarrollan. Empero, se debe tener en cuenta que, a causa de la naturaleza agregada de los datos, las conclusiones derivadas del modelo estadístico tienen que ser matizadas para evitar caer en una falacia ecológica. Es decir, las inferencias a nivel agregado no se pueden trasladar directamente a conclusiones sobre los individuos. No obstante lo anterior, los resultados pueden y deben considerarse a la luz de otras investigaciones que, estas sí, permitan obtener conclusiones a nivel individual.\\ 

Ahora bien, debido a que este trabajo es un proyecto de titulación para dos carreras pertenecientes a áreas del conocimiento distintas, decidí estructurarlo en 3 apartados relativamente independientes entre sí. Esto quiere decir que dependiendo del interés de cada lector, es posible concentrarse en cada uno de ellos por separado, estudiarlos en un orden diferente al presentado u omitir alguna de las 3 partes.\\ 

En primer lugar, considero que cualquier análisis estadístico aplicado tiene que estar acompañado de un conocimiento general del problema en cuestión. Es necesario entonces contar con un marco teórico sobre los movimientos NAP en general y el Front National en particular. Este es el objetivo de la primera parte. El capítulo 1 define en términos más formales a los movimientos NAP y presenta una primera referencia de lo que caracteriza al FN como un partido de dicha corriente política. El capítulo 2 tiene como objetivo familiarizar al lector con el sistema político francés así como con la historia del FN dentro de dicho sistema. El tercer capítulo, por su parte, recoge una serie de teorías presentes en la literatura sobre las razones por las cuales las personas votan por estos movimientos y partidos. Es con relación a estas teorías que busqué obtener las variables para el modelo estadístico.\\

La segunda parte de la tesis constituye el marco teórico estadístico en el que baso el análisis de los datos franceses. Así como es importante conocer el contexto al que aplicaremos la estadística--- nuestro patio de juego--- es fundamental el contexto estadístico por sí mismo, sobre todo en una tesis de Actuaría. Así pues, esta segunda parte me permitirá explorar un paradigma estadístico no siempre visto en la licenciatura, pero que tuve la fortuna de conocer a través de materias optativas: la estadística bayesiana.\\ 

En este sentido, el capítulo 4 es una introducción a dicho paradigma. En él presento la interpretación bayesiana de la probabilidad como medida de incertidumbre y no solo de variabilidad, comparándola con las más conocidas interpretaciones clásica y frecuentista. Asimismo, menciono algunas estrategias para asignar distribuciones iniciales que puedan ser actualizadas a la luz de los datos mediante el proceso de aprendizaje bayesiano y así analizar los fenómenos de interés con base en la distribución posterior.\\ 

El capítulo 5, por su parte, constituye una aproximación teórica a los modelos objeto de esta tesis. Comienzo exponiendo el modelo de regresión lineal y continúo presentando la regresión logística como un caso particular de un modelo lineal generalizado. Finalmente discuto el modelado jerárquico o multinivel como importante alternativa a los más conocidos enfoques de regresión cuando se están estudiando fenómenos con subpoblaciones provenientes de una misma población general. En lugar de realizar una agrupación completa de los datos mediante una sola regresión poblacional o de presentar tantas regresiones independientes como subpoblaciones hayan, el modelado jerárquico busca realizar una agrupación parcial a través del concepto de intercambiabilidad.\\

Debido a las implicaciones que conlleva la perspectiva bayesiana, el capítulo 6 presenta algunos de los métodos computacionales que permiten llevar a cabo inferencias estadísticas bayesianas en la práctica. De manera particular, recorro un poco la historia y características de los dos algoritmos más conocidos de MCMC: Metropolis-Hastings y \textit{Gibbs Sampler}. Este recorrido busca llevar al lector a entender de mejor manera un atractivo método de MCMC, relativamente más reciente y desconocido, conocido como \textit{Hamiltonian Monte Carlo} y que será el utilizado en la última parte de la tesis mediante un \textit{software} de vanguardia llamado Stan.\\

Finalmente, la tercera parte de la tesis constituye propiamente el modelado de los datos franceses. Una vez que se cuenta con ambos marcos teóricos--- el cualitativo y el cuantitativo--- procedo al análisis. El capítulo 7 presenta los datos de manera descriptiva y mediante un análisis exploratorio. El capítulo 8 es el proceso de modelado en sí: contiene los diferentes modelos aplicados a los datos franceses y la discusión de por qué fueron considerados hasta llegar a un modelo final con base en una medida de pérdida esperada llamada WAIC, así como la interpretación de los resultados de dicho modelo final. Por último, el noveno capítulo contiene las conclusiones generales del proyecto.\\ 

Debido a las características de los gráficos presentados en esta tercera parte, así como el desarrollo del análisis a través del lenguaje $\mathsf{R}$ y de Stan, recomiendo al lector consultar la versión digital de esta tesis, disponible en el repositorio \url{https://github.com/fazepher/Tesis-ITAM}. Es a este vínculo al que me refiero cuando en el texto mencione el \textit{repositorio de esta tesis}.
