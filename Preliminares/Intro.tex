\chapter*{Introducción}
\addcontentsline{toc}{chapter}{Introducción}

Al cursar la Licenciatura en Actuaría en el ITAM, elegí el área de concentración en estadística. Encontré que esta es el puente perfecto entre mi inclinación cuantitativa y mi gusto por las ciencias sociales. Sin perder formalidad y rigor matemáticos, resulta ser ampliamente aplicable a problemas reales. Bien dijo John Tukey que la mejor parte de ser un estadístico es que puedes jugar en el patio trasero de todo mundo; el mío son los fenómenos políticos.\\

Además de Actuaría también cursé en el ITAM la Licenciatura en Relaciones Internacionales. Por ello, me resultó natural presentar como tesis de ambas carreras un estudio estadístico relacionado con uno de los fenómenos políticos internacionales que más han despertado interés en los últimos años: aquellos movimientos, desarrollados particularmente en Europa, que usualmente son llamados de \textit{derechas extremas}.\\ 

Ni la academia ni los medios de comunicación logran un consenso sobre cómo llamar o clasificar a esta familia política. Es posible encontrar etiquetas como \textit{derechas radicales}, \textit{nacionalismo populista}, \textit{populismos de derecha}, \textit{tribalismo reaccionario}, \textit{neopopulismo} o, incluso, \textit{neofascismo}, \textit{postfascismo} o \textit{neonazismo} \parencites{Mudde07a}{Mammone12}{Hainsworth16a}. A pesar de ello, de acuerdo con Cas Mudde, es posible definir a la familia política con base en tres características básicas \parencites{Mudde07a}{Beauchamp16a}. En primer lugar, el nativismo como forma xenófoba de nacionalismo. En segundo lugar, son movimientos autoritarios, que privilegian el discurso de la seguridad, la ley y el orden. Finalmente, comparten el populismo como forma no solo de hacer política sino, más fundamentalmente, de ver a la sociedad: frente a las élites corruptas--- \textit{el ellos}--- el pueblo puro--- \textit{el nosotros}---. Mudde los llama movimientos de derecha radical populistas aunque en esta ocasión los llamaré NAP--- nativistas autoritarios de corte populista--- con el objetivo de resaltar sus características fundamentales.\\

De manera particular, el partido que pudiera considerarse \textit{pater familias} de esta corriente es el Front National francés \parencite{Mudde07a}. Fundado, entre otros personajes, por Jean-Marie Le Pen en 1972 \parencite{Stockemer17}, surgió en la escena política en 1984 al obtener algo más de 2 millones de sufragios en las elecciones europeas de ese año \parencite{LeBras15}. En 2002, Le Pen avanzó de manera sorpresiva a la segunda vuelta presidencial, aunque finalmente perdió frente a Jacques Chirac. En los últimos años, ya con Marine Le Pen--- hija del fundador--- como lideresa, el FN ha roto el tradicional sistema bipartidista de Francia \parencite{LeBras16};  en las elecciones presidenciales de 2017 ella también avanzó a la segunda vuelta presidencial.\\

Ante esta situación, las preguntas aparecen. ¿Por qué la gente vota por estas propuestas? ¿Qué tipo de votante los ha apoyado y qué tipo de votante los rechaza? ¿Dónde han tenido más votos? ¿Cuáles son los terrenos fértiles que han permitido este surgimiento y cuáles han fungido como barreras a su crecimiento? Esta tesis pretende contribuir al entendimiento de este fenómeno mediante un estudio de caso.\\ 

A partir de datos agregados buscaré describir las \textit{configuraciones sociales} en las se desarrolló el FN en las 4 elecciones de 2007 y 2012. Debido a la naturaleza agregada de los datos, cualquier conclusión derivada del modelado estadístico de los mismos tiene que ser matizada para evitar caer en una falacia ecológica. Sin embargo, el presente estudio puede considerarse en conjunto con otras investigaciones que permiten obtener conclusiones a nivel individual {\color{Red} (Joël Gombin)}.\\ 

Debido a que este trabajo es un proyecto de titulación para dos carreras pertenecientes a áreas del conocimiento distintas, decidí estructurarlo en 3 apartados relativamente independientes pero complementarios entre sí. Esto quiere decir que dependiendo del interés de cada lector, es posible concentrarse en cada uno de ellos por separado, estudiarlos en un orden diferente al presentado u omitir alguna de las 3 partes.\\ 

En primer lugar, considero que cualquier análisis estadístico aplicado tiene que estar acompañado de un conocimiento general del problema en cuestión. Es necesario entonces contar con un marco teórico sobre los movimientos NAP en general y el Front National en particular. Este es el objetivo de la primera parte. El capítulo 1 define en términos más formales a los movimientos NAP y presenta una primera referencia de lo que caracteriza al FN como un partido de dicha corriente política. El capítulo 2 tiene como objetivo familiarizar al lector con el sistema político francés así como con la historia del FN dentro de dicho sistema. El tercer capítulo, por su parte, recoge una serie de teorías presentes en la literatura sobre las razones por las cuales las personas votan por los movimientos NAP. Es con relación a estas teorías que busqué obtener las variables del modelo estadístico.\\

La segunda parte de la tesis constituye el marco teórico estadístico en el que baso el análisis de los datos franceses. Así como es importante conocer el contexto al que aplicaremos la estadística--- nuestro patio de juego--- es fundamental el contexto estadístico por sí mismo. En este sentido, el capítulo 4 representa una introducción al paradigma estadístico bayesiano bajo el cual trabajo. El capítulo 5, por su parte, constituye una aproximación teórica a los modelos objeto de esta tesis. Debido a las implicaciones prácticas que conlleva modelar desde la perspectiva bayesiana, el capítulo 6 recorre los métodos computacionales que permiten llevar a cabo inferencias estadísticas bayesianas.\\

Finalmente, la tercera parte de la tesis es el modelado de los datos franceses. Una vez que se cuenta con ambos marcos teóricos--- el cualitativo y el cuantitativo--- procedo al análisis. El capítulo 7 presenta los datos de manera descriptiva y mediante un análisis exploratorio. El capítulo 8 contiene el modelado en sí: los diferentes modelos aplicados a los datos franceses y la discusión de por qué fueron considerados. Por último, el noveno capítulo contiene la interpretación de los resultados y las conclusiones generales del proyecto. 
