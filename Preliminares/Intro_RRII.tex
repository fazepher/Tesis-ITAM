\chapter*{Introducción}
\addcontentsline{toc}{chapter}{Introducción}

En el ITAM, además de estudiar la Licenciatura en Relaciones Internacionales, realicé la carrera simultánea de Actuaría. Así pues, inicié como tesis de ambas carreras\footnote{Debido a los requisitos de extensión de los trabajos de titulación en el Departamento Académico de Estudios Internacionales, esta es una versión más corta y ligeramente adaptada de la tesis conjunta completa, misma que puede consultarse en \textcite{TesisAct} así como en el repositorio abierto de la tesis, disponible en \url{https://github.com/fazepher/Tesis-ITAM}. Considerando las características de los gráficos presentados en el trabajo, la versión digital puede resultar útil para el lector.} un estudio estadístico relacionado con uno de los fenómenos políticos internacionales que más han despertado interés en los últimos años: aquellos movimientos populistas que usualmente son llamados de \textit{derechas extremas o radicales}.\\  

Desarrollada particularmente en Europa, ni la academia ni los medios de comunicación logran un consenso sobre cómo llamar o clasificar a esta familia política de derechas extremas. A pesar de ello, de acuerdo con Cas Mudde, es posible definir a la familia política con base en tres características básicas \parencites{Mudde07a}{Beauchamp16a}. En primer lugar, comparten el nativismo como forma xenófoba de nacionalismo. En segundo lugar, son movimientos autoritarios, que privilegian el discurso de la seguridad, la ley y el orden. Finalmente, los une el populismo como forma no solo de hacer política sino, más fundamentalmente, de ver a la sociedad: frente a las élites corruptas--- \textit{el ellos}--- el pueblo puro--- \textit{el nosotros}---. Mudde los llama movimientos de derecha radical populistas aunque aquí los llamaré NAP--- nativistas autoritarios de corte populista--- con el objetivo de resaltar sus características fundamentales.\\

De manera particular, el partido que pudiera considerarse \textit{pater familias} de esta corriente es el \textit{Front National} francés \parencite{Mudde07a}. Este fue fundado, entre otros personajes, por Jean-Marie Le Pen en 1972. Surgió en la escena política en 1984 al obtener algo más de 2 millones de sufragios en las elecciones europeas de ese año \parencite{LeBras15}. En 2002, Le Pen avanzó de manera sorpresiva a la segunda vuelta presidencial, aunque finalmente perdió frente a Jacques Chirac. En los últimos años, ya con Marine Le Pen--- hija del fundador--- como lideresa, el FN ha roto el bipartidismo en Francia \parencite{LeBras16};  en las elecciones presidenciales de 2017 ella también avanzó a la segunda vuelta presidencial. En junio de 2018 el partido cambió oficialmente de nombre a \textit{Rassemblement National} con miras a seguir creciendo en medio de la ola nativista que pareciera darse en el mundo.\\

Las preguntas respecto a los movimientos NAP son múltiples. ¿Por qué la gente vota por estas propuestas, siendo que algunas son abiertamente racistas? ¿Qué tipo de votante los ha apoyado y cuál los rechaza? ¿Dónde han tenido más votos? ¿Cuáles son los terrenos fértiles que han permitido este surgimiento y cuáles han fungido como barreras a su crecimiento? De hecho, es la familia política más estudiada en los últimos años \parencite{Mudde16}.\\ 

Siendo el referente tradicional de los movimientos NAP, muchos libros y artículos se han publicado en la búsqueda por entender al votante y la sociología política del Front National. La mayoría de ellos se aproximan a las preguntas mediante estudios cualitativos o con base en datos individuales provenientes de encuestas de representatividad nacional.\\ 

Esta tesis pretende complementar el entendimiento del fenómeno mediante un estudio de caso, aunque con una estrategia distinta. A partir de datos censales agregados--- también llamados ecológicos--- y técnicas de modelado estadístico jerárquico, buscaré describir las \textit{configuraciones sociales} que favorecieron o inhibieron el voto por la candidatura frontista de Marine Le Pen en la elección presidencial de 2012. Es decir, ¿acaso las zonas de mayor desempleo votaron en mayor medida por Le Pen?, ¿fueron más bien los lugares de mayor inmigración los que apoyaron a la candidata frontista?, ¿lugares con mayor presencia de personas escolarizadas experimentaron menores niveles de apoyo al FN?\\

Este enfoque de modelado jerárquico de datos ecológicos, tiene la ventaja, como bien apunta Joël \textcite{Gombin05}, de reconocer y modelar la variabilidad territorial de los fenómenos políticos, así como la importancia del contexto social en el que los individuos se desarrollan. Empero, se debe tener en cuenta que, a causa de la naturaleza agregada de los datos, las conclusiones derivadas del modelo estadístico tienen que ser matizadas para evitar caer en una falacia ecológica. Es decir, las inferencias a nivel agregado no se pueden trasladar directamente a conclusiones sobre los individuos. No obstante lo anterior, los resultados pueden y deben considerarse a la luz de otras investigaciones que, estas sí, permitan obtener conclusiones a nivel individual.\\ 

Ahora bien, debido a que este trabajo es un proyecto de titulación para dos carreras pertenecientes a áreas del conocimiento distintas, los capítulos pueden considerarse estructurados en 3 partes relativamente independientes entre sí. Dependiendo del interés de cada lector, es posible concentrarse en cada una de ellas por separado o estudiarlas en un orden diferente.\\ 

En primer lugar, es necesario contar con un marco teórico sobre los movimientos NAP en general y el Front National en particular. Este es el objetivo de la primera parte. El capítulo 1 define en términos más formales a los movimientos NAP y presenta una primera referencia de lo que caracteriza al FN como un partido de dicha corriente política. El capítulo 2 tiene como objetivo familiarizar al lector con el sistema político francés así como con la historia del FN dentro de dicho sistema. El tercer capítulo, por su parte, recoge una serie de teorías presentes en la literatura sobre las razones por las cuales las personas votan por estos movimientos y partidos. Es con relación a estas teorías que busqué obtener las variables para el modelo estadístico.\\

El cuarto capítulo, en esta versión del documento conjunto, constituye la segunda parte de la tesis. En él exploro el marco teórico estadístico sobre el cual baso el análisis de los datos franceses. En particular presento un paradigma estadístico que considero particularmente útil para las ciencias sociales: la inferencia bayesiana. De manera sucinta podemos decir que la interpretación bayesiana de la probabilidad es la de una medida de incertidumbre y no solo de variabilidad, en contraposición con las más conocidas interpretaciones clásica y frecuentista. El lector interesado en una discusión más profunda del contexto estadístico puede consultar el desarrollo completo de la tesis en el repositorio o en la versión impresa \parencite{TesisAct}.\\

Finalmente, la tercera parte de la tesis es propiamente el modelado de los datos franceses. Una vez que se cuenta con ambos marcos teóricos--- el cualitativo y el cuantitativo--- procedo al análisis. El capítulo 5 presenta los datos de manera descriptiva y mediante un análisis exploratorio. El capítulo 6 es el proceso de modelado en sí: contiene el modelo final aplicado a los datos franceses así como la interpretación de los resultados de dicho modelo. Por último, el séptimo capítulo contiene las conclusiones generales del proyecto.\\ 
